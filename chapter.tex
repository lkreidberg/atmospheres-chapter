%%%%%%%%%%%%%%%%%%%% author.tex %%%%%%%%%%%%%%%%%%%%%%%%%%%%%%%%%%%
%
% template for chapters to the Handbook of Exoplanets
% modified by H. Deeg from the 'template.tex' provided by Springer for the svmult.cls class
% 20Mar 2016
%
%%%%%%%%%%%%%%%% Springer %%%%%%%%%%%%%%%%%%%%%%%%%%%%%%%%%%


% RECOMMENDED %%%%%%%%%%%%%%%%%%%%%%%%%%%%%%%%%%%%%%%%%%%%%%%%%%%
\documentclass[graybox,natbib,nosecnum]{svmult}
\bibpunct{(}{)}{;}{a}{}{,} % suppress commas between author-names and year

\pdfoutput=1   %forces use of pdflatex. Disable if you prefer to use .eps or .ps figures.
% choose options for [] as required from the list
% in the Reference Guide

\usepackage{mathptmx}       % selects Times Roman as basic font
\usepackage{helvet}         % selects Helvetica as sans-serif font
\usepackage{courier}        % selects Courier as typewriter font
\usepackage{type1cm}        % activate if the above 3 fonts are
                            % not available on your system

\usepackage{makeidx}         % allows index generation
\usepackage{graphicx}        % standard LaTeX graphics tool
                             % when including figure files
\usepackage{multicol}        % used for the two-column index
\usepackage[bottom]{footmisc}% places footnotes at page bottom
\usepackage[normalem]{ulem}	% for strike-through of text with \sout{}  
\usepackage{hyperref}  %for hyperlinks

\usepackage{soul}   % for high-lighting of text
% see the list of further useful packages
% in the Reference Guide

% expansions of  journal abbreviations from bibtex entries by ADS
% adapted to Springer Basic style (no periods in abbreviations)
\newcommand*\aap{A\&A}
\let\astap=\aap
\newcommand*\aapr{A\&A Rev}
\newcommand*\aaps{A\&AS}
\newcommand*\actaa{Acta Astron}
\newcommand*\aj{AJ}
\newcommand*\ao{Appl Opt}
\let\applopt\ao
\newcommand*\apj{ApJ}
\newcommand*\apjl{ApJ}
\let\apjlett\apjl
\newcommand*\apjs{ApJS}
\let\apjsupp\apjs
\newcommand*\aplett{Astrophys Lett}
\newcommand*\apspr{Astrophys Space Phys Res}
\newcommand*\apss{Ap\&SS}
\newcommand*\araa{ARA\&A}
\newcommand*\azh{AZh}
\newcommand*\baas{BAAS}
\newcommand*\bac{Bull astr Inst Czechosl}
\newcommand*\bain{Bull Astron Inst Netherlands}
\newcommand*\caa{Chinese Astron Astrophys}
\newcommand*\cjaa{Chinese J Astron Astrophys}
\newcommand*\fcp{Fund Cosmic Phys}
\newcommand*\gca{Geochim Cosmochim Acta}
\newcommand*\grl{Geophys Res Lett}
\newcommand*\iaucirc{IAU Circ}
\newcommand*\icarus{Icarus}
\newcommand*\jcap{J Cosmology Astropart Phys}
\newcommand*\jcp{J Chem Phys}
\newcommand*\jgr{J Geophys Res}
\newcommand*\jqsrt{J Quant Spectr Rad Transf}
\newcommand*\jrasc{JRASC}
\newcommand*\memras{MmRAS}
\newcommand*\memsai{Mem Soc Astron Italiana}
\newcommand*\mnras{MNRAS}
\newcommand*\na{New A}
\newcommand*\nar{New A Rev}
\newcommand*\nat{Nature}
\newcommand*\nphysa{Nucl Phys A}
\newcommand*\pasa{PASA}
\newcommand*\pasj{PASJ}
\newcommand*\pasp{PASP}
\newcommand*\physrep{Phys Rep}
\newcommand*\physscr{Phys Scr}
\newcommand*\planss{Planet Space Sci}
\newcommand*\pra{Phys Rev A}
\newcommand*\prb{Phys Rev B}
\newcommand*\prc{Phys Rev C}
\newcommand*\prd{Phys Rev D}
\newcommand*\pre{Phys Rev E}
\newcommand*\prl{Phys Rev Lett}
\newcommand*\procspie{Proc SPIE}
\newcommand*\qjras{QJRAS}
\newcommand*\rmxaa{Rev Mexicana Astron Astrofis}
\newcommand*\skytel{S\&T}
\newcommand*\solphys{Sol Phys}
\newcommand*\sovast{Soviet Ast}
\newcommand*\ssr{Space Sci Rev}
\newcommand*\zap{ZAp}


\newcommand{\hbindex}[1]{\hl{#1}\index{#1}}  %highlights index entries

\newcommand{\project}[1]{\textsl{#1}}
\newcommand{\JWST}{\project{JWST}}
\newcommand{\HST}{\project{HST}}
\newcommand{\TESS}{\project{TESS}}
\newcommand{\Spitzer}{\project{Spitzer}}
\newcommand{\Kepler}{\project{Kepler}}


\makeindex             % used for the subject index
                       % please use the style svind.ist with
                       % your makeindex program

%%%%%%%%%%%%%%%%%%%%%%%%%%%%%%%%%%%%%%%%%%%%%%%%%%%%%%%%%%%%%%%%%%%%%%%%%%%%%%%%%%%%%%%%%

\begin{document}

\title*{Exoplanet Atmosphere Measurements from Transmission Spectroscopy and other Planet-Star Combined Light Observations}
\titlerunning{Atmosphere Measurements from Combined Light Observations} 
% Use \titlerunning{Short Title} for an abbreviated version of
% your contribution title if the original one is too long
\author{Laura Kreidberg}
% Use \authorrunning{Short Title} for an abbreviated version of
% your contribution title if the original one is too long
\institute{Laura Kreidberg \at Harvard Society of Fellows, 78 Mount Auburn Street, Cambridge, MA, 02138, \email{laura.kreidberg@cfa.harvard.edu}}
%
% Use the package "url.sty" to avoid
% problems with special characters
% used in your e-mail or web address
%
\maketitle


\abstract{Abstract.}
%This document is intended as a template and guide for the preparation of chapters for the Handbook of Exoplanets, using latex. It complements the `Guidelines for Authors of the Exoplanet Handbook', which are distributed as a separate document. This template was adapted from Springer's template `author.tex' for contributed books, using the style svmult.cls (distributed together with this template). \\ Each chapter should be preceded by an abstract (10--15 lines long) that summarizes the content. The abstract will appear \textit{online} at \url{www.SpringerLink.com} and at ADS and be available with unrestricted access. This allows unregistered users to read the abstract as a teaser for the complete chapter. For the title you are encouraged to follow the tips for Search Engine Optimization given below.}


\section{Introduction }
%"Don't you just hate it when you travel 1,200 light years to a planet scientists have assured you is "Earthlike," and you get there, and there's, like, NO ATMOSPHERE?!!" -- Martin BG, Potomac, MD



\section{Observing Techniques} 
In this chapter we will focus on exoplanet atmosphere characterization using the combined light from the planet and its host star. In contrast to direct imaging, which seeks to mask the stellar flux, the combined light approach capitalizes on the star as a constant reference point. As the planet moves through its orbit, it absorbs, re-emits, and reflects light from its host star. These planetary signals can be measured differentially relative to the baseline stellar flux.

\subsection{Transit Spectroscopy}
The most widely used combined-light technique is transmission spectroscopy. For this method, the planet is observed in transit as it passes in front of its host. The resulting time series of brightness measurements is known as the transit light curve. During the transit, the planet blocks a small fraction of the stellar flux equal to the sky-projected area of the planet relative to the area of the host star (as illustrated in Figure FIXME). We refer to this fraction as the \hbindex{transit depth}, $\delta = (R_p/R_s)^2$. 

The key idea behind transmission spectroscopy is that the apparent size of the planet is not actually a single value: rather, it depends on the wavelength of light used for the observation.  At wavelengths where the atmosphere is relatively more opaque due to absorption by atoms or molecules, the planet blocks slightly more stellar flux.  To measure these variations, the light curve is binned in wavelength into spectrophotometric channels, and the light curve from each channel is fit separately with a transit model (see section FIXME).  The measured transit depths as a function of wavelength constitute the \hbindex{transmission spectrum}, so named because it arises from the transmission of stellar flux through the planet's atmosphere.
 
Theoretical models for the transmission spectrum require radiative transfer calculation for light on the slant path through the atmosphere \citep{seager00}, a computationally intensive task. However, we can make a rough prediction of the size of features in the transmission spectrum based on the atmospheric \hbindex{scale height} $H$. The scale height is the change in altitude over which the pressure drops by a factor of $e$. Assuming hydrostatic equilibrium, 

\begin{equation}
H = \frac{K_bT_{eq}}{\mu g}
\end{equation}
where $K_b$ is the Boltzmann constant, $T_{eq}$ is the planet's \hbindex{equilibrium temperature}, $\mu$ is the mean molecular mass, and $g$ is the surface gravity (FIXME). 

The amplitude of spectral features in transmission is then:
\begin{eqnarray}
\delta_\lambda &=& \frac{(R_p + nH)^2}{R_s^2} - \frac{R_p^2}{R_s^2}\\
 & \approx & 2R_pH/R_s^2
\end{eqnarray} 
where $n$ is the number of scale heights crossed at wavelengths with high opacity (typically around two for cloud-free atmospheres, \cite{stevenson16}. The ideal candidates for transmission spectroscopy is a planet with high equilibrium temperature, a small host star, low surface gravity, and a low mean molecular mass atmosphere. Even for the best cases, the amplitude of spectral features is just $\delta_\lambda \sim0.1\%$ \citep[e.g. WASP 121-b;][]{evans16}. For Earth-like planets, the expected amplitude is two to three orders of magnitude smaller, depending on host star size. 

%It should be emphasized how miniscule these variations in transit depth are. Even for the best case scenario -- a hot, hydrogen-rich atmosphere with low surface gravity -- the amplitude of features is 

%Transmission spectrum measurements therefore require extraordinary care and investment of telescope resources, as discussed in FIXME.

\subsection{Occultation Spectroscopy}
A close cousin of the transit spectroscopy method is \hbindex{occultation spectroscopy}, which measures thermal emission and reflection from the planet. Rather than observing the planet during transit, it is observed at secondary eclipse, when it passes behind the host star. The eclipse provides a baseline measurement of the stellar flux alone. When the planet rotates back into view, any increase in brightness can be attributed to the planet's thermal emission and reflected light.

For the short-period planets that have been studied so far, the dominant source of thermal emission is re-radiation of incident stellar flux (rather than latent heat of formation, as seen for directly imaged planets). Thus the typical size of the emission signal can be predicted from the planet's equilibrium temperature:

\begin{equation}
\label{eqn:fpfs}
\frac{F_p}{F_s} = \frac{B(\lambda, T_{eq})}{B(\lambda, T_s)}\left(\frac{R_p}{R_s}\right)^2
\end{equation}
where $F_p/F_s$ is the planet-to-star flux ratio, $B(\lambda, T)$ is the blackbody spectral radiance at temperature $T$, and $R_p/R_s$ is the planet-to-star radius ratio. Since the planet is cooler than the star, the flux ratio is larger at longer wavelengths. For example, the planet-to-star flux for the hot Jupiter HD 209458b is just 50 parts per million at 1 $\mu$m, but increases to over 1000 ppm at 4.5 $\mu$m \citep{line16}.

Equation \ref{eqn:fpfs} is a good first order approximation of the planet signal, but as with transmission spectroscopy, more complex features arise in the spectrum due to the atmosphere's changing opacity with wavelength. The emitted light comes from the photosphere of the planet, where the optical depth is of order unity.  As opacity increases, so does the altitude of the photosphere.  Therefore, since temperature changes with altitude, the observed atmosphere may be above or below $T_{eq}$, depending on the wavelength of the measurement and the altitude it probes. 

%A useful way to describe the temperature of the photosphere is the \hbindex{brightness temperature}: the temperature a blackbody needed to replicate the observed spectral intensity at a given wavelength.

At short wavelengths, reflected light from the planet may also be detectable. It is convenient to quantify the amount of reflected light in terms of a ``perfect" mirror: a flat, reflecting disk with the same cross-sectional area as the planet. The ratio of reflected light from the fully illuminated planet, relative to reflection by a perfect mirror, is the geometric albedo $A_g$. The total reflected light signal is

\begin{equation}
F_{reflect} = A_g(R_p/a)^2 \Phi(\alpha)
\end{equation}
where $a$ is the orbital separation and $\Phi(\alpha)$ is the phase function (the reflected light intensity at phase angle $\alpha$). The phase function depends on the scattering properties of the atmosphere, but analytic predictions are available for certain simplified models \citep[e.g.][]{madhu12}.  Reflected light is easiest to detect in the optical, where it dominates the thermal emission signal; however, the amplitude tends to be small \citep[typically less than 100 ppm][]{angerhausen15}.

\runinhead{Phase Curves}
Short-period planets that are tidally locked to their host stars pose a unique opportunity for thermal emission and reflected light measurements. The rotation period for these planets is known (equal to their orbital period), so over the course of one complete orbit all longitudes are visible in turn. A \hbindex{phase curve} observation consists of continuous time series photometry or spectroscopy of a planet over its orbit, using the stellar flux at secondary eclipse as a baseline. These observations are powerful in that they probe atmospheric physics and chemistry over a wider geographic region than transits or eclipses alone; however, they are challenging from a technical standpoint because they require long stares at a single object. %In principle it is possible to apply this technique to longer period planets that are not tidally locked; however, in addition to the feasibility challenge the interpretation is also problematic. 

\runinhead{Eclipse Mapping}
All of the occultation spectroscopy techniques described above rely on a hemisphere average of the planet's reflected or emitted light. To glean additional spatial information, one can use the \hbindex{eclipse mapping} method, which relies on precise observations of secondary eclipse ingress and egress.  During these intervals, the hemisphere of the planet is partially eclipsed by the star, allowing the observer to pinpoint the brightness of a smaller region and thus map the brightness distribution in detail \citep{rauscher07}. In particular, for planets with a nonzero impact parameter, the eclipse mapping technique is sensitive to changes in brightness with latitude (as higher latitudes are first to enter and exit eclipse). 
% This approach is based on the fact that planets are not uniformly bright. In the case of a hot gas giant, the hottest part of the atmosphere may be offset from the substellar point due to atmospheric circulation. For a terrestrial planet, continents and oceans with varying reflectivity may be present. 

\subsection{Ground-Based}
The observing techniques described above are best suited to space telescopes: Earth's atmosphere is an impediment to measuring precise light curves because its transparency changes with time.  As a target moves through the sky, line-of-sight properties such as the airmass, precipitable water vapor, and cloud coverage will vary, introducing time-correlated red noise in the light curve that may be orders of magnitude larger than the signal from the planet. Despite this challenge, there are several inventive strategies that have made it possible to characterize exoplanet atmospheres from the ground. 

\runinhead{Photometry and Multi-Object Spectrographs}
One approach is to use comparison stars to correct for systematic trends in the target light curve. For photometry, this is a straightforward process: in addition to the target, several nearby stars are observed in the same field of view. The path of their flux traverses similar parts of the Earth's atmosphere, so their light curves exhibit close-to-identical systematic trends as the target. The target light curve can then be divided by the sum of the comparison star light curves to remove systematics.  %The same principle applies to spectroscopic observations, but requires a wide slit ($>10''$) to avoid loss of light due to time variable seeing \citep{bean10}. Custom-made slit masks are available for several multi-object spectrographs, including Magellan/LDSS3c, Gemini/GMOS, Keck/MOSFIRE?, FIXME.

The same principle applies to spectroscopic observations, but requires a more specialized approach.  Spectrographs usually image the target through a slit prior to dispersing the light, in order to block out contamination from the sky background and any nearby stars. However, the slits are typically narrower than the seeing disk of the target star, and some fraction of flux -- highly dependent on the seeing -- is outside the slit.  To avoid these slit losses, one can use a multi-object spectrograph with a custom-made mask that has wide slits ($>10''$), a technique developed by \cite{bean10}. This method produces close to space-telescope quality precision \citep[e.g.][]{}.  There are several multi-object spectrographs available with a custom slit mask option, including Magellan/LDSS3c, Gemini/GMOS, Keck/, FIXME.  

\runinhead{High Resolution Spectrographs}
Another creative ground-based technique uses high resolution spectroscopy to detect the planet's atmosphere.  The idea is that the planet spectrum is Doppler shifted due to its orbital motion, and thus separated in wavelength from both the stellar spectrum and telluric lines from the Earth's atmosphere. The planet spectrum can be cross-correlated with a template to reveal the atmospheric composition and orbital velocity \citep[e.g.][]{snellen10}.  See Chapter FIXME for more on high resolution observations. %This technique requires a high resolution spectrograph to separate individual spectral lines ($R \sim 10^5$) and high SNR spectra (FIXME: how high?), so is only possible with a few current facilities. The most widely used instrument is VLT/CRIRES (FIXME), and there have also been detections with Keck/NIRSPEC and HARPS \citep{lockwood14, martins15}. 

\section{Observing Facilities}
Detecting the tiny signals from exoplanet atmospheres is a formidable challenge. Even for the most favorable systems, the amplitude of spectral features is of order a tenth of a percent \citep[e.g.][]{evans16}. Pushing to this level of precision requires many photons (and thus large telescopes) as well as a stable observing environment (ideally in space).  In this section, we summarize the observing facilities that are best suited to this task (current as of January 2017).

\subsection{Space-Based}
Observing exoplanets from space is advantageous in several ways. First, space telescopes are free from atmospheric turbulence that adds systematic noise to light curves (see Section FIXME for more on red noise). In addition, wavelength regions that are blocked by the Earth's atmosphere become accessible (e.g., ultra-violet and water absorption bands).  Space-based detectors can also be cooled to low temperatures, enabling longer wavelength observations than are possible from the ground.

\runinhead{Hubble Space Telescope}
The 2.7 m \emph{Hubble Space Telescope} (\HST) has two instruments widely used for atmosphere studies: the Wide Field Camera 3 (WFC3) and the Space Telescope Imaging Spectrograph (STIS). 

WFC3 is \HST's newest detector (installed in 2009) and succeeds NICMOS (the Near Infrared Camera and Multi-Object Spectrometer). It provides near-infrared, low-resolution spectroscopy from $0.8 - 1.7$ $\mu$m, where water molecules have strong absorption features (FIXME ref).  As of 2017, WFC3 measurements are the gold standard in atmosphere characterization: they are photon-limited, highly repeatable, and precise \citep[e.g.][]{deming13, kreidberg14a}. The best achieved precision to date is 15 ppm on the eclipse depths for HD 209458b at a resolution $R\sim 30$ \citep{line16}. FIXME NICMOS?

The other \HST\ instrument frequently used for exoplanet observations is STIS, a UV/optical instrument sensitive to the wavelength range $0.1 - 1 \, \mu$m. This range covers absorption lines from atomic alkali metals (e.g., sodium, potassium, magnesium), as well as the Lyman-$\alpha$ transition of atomic hydrogen. 

\runinhead{Spitzer Space Telescope}
In addition to \HST, the other space-based observing facility is \Spitzer, an 85 cm telescope launched in 2003. \Spitzer\ provided infrared spectroscopy from 3 - 160 $\mu$m until 2009, when it ran out of coolant. Now operating in its Warm Mission, \Spitzer\ is currently capable of broadband photometry at 3.6 and 4.5 $\mu$m.  \Spitzer's unique strength is its long wavelength coverage, where planet-to-star flux ratios are higher. It is therefore often used for thermal emission measurements. \Spitzer\ also provides continuous observing capability thanks to its heliocentric orbit, and has led the way in observations of thermal phase curves \citep[e.g.][]{knutson07}. 


\section{Best Practices for Achieving Precise and Accurate Measurements}
The basic format for observing transiting planets is time series spectroscopy or photometry, with the goal of detecting small deviations in brightness over time due to the planet's orbital motion.  In this section we summarize a few good p and discuss sources of error (both astrophysical and instrumental).


\subsection{Observational Design}
Designing a successful observation is in some ways more of an art than a science -- no currently available instruments were built with transiting planets in mind, and the planet signal is often orders of magnitude smaller than the typical precision of the instrument (FIXME ref).  But there are a few general principles that can increase the probability of success, which we summarize here:

\begin{itemize}
\item{\emph{The more observations the better.}}
\item{\emph{Observe enough baseline.} All combined planet/star observations rely on differential brightness measurements -- i.e., the flux measured during transit or eclipse \emph{relative} to the brightness out of transit/eclipse. Based on photon noise alone, the minimum amount of time spent observing baseline should equal the transit/eclipse duration. For the case of short durations, one hour before and after the transit  allow }
\item{\emph{Avoid pushing the detector to its limit.} The best targets for atmosphere characterization orbit bright stars, and a common temptation is to lengthen exposure times to collect as many of those photons as possible. However, as the detector approaches saturation its behavior is harder to  }
\end{itemize}

\subsection{Light Curve Fitting}
The reduced data consist of a time series of brightness measurements (either a ``white" light curve summed up over the entire bandpass, or spectroscopic light curves created by binning the spectrum into channels).  The most general model for the data is:

\begin{equation}
F(t) = S(t) \times T(t) \times E(t) \times \phi(t)
\end{equation}
where $S(t)$ is the systematics model (for signals that are not astrophysical; e.g., a linear trend with time), $T(t)$ is the transit model, $E(t)$ is the eclipse, and $\phi(t)$ is the phase variation from reflected and emitted light. In cases where only a transit or eclipse is observed, the phase variation can usually be neglected.

Transit and eclipse models both depend on the planet's orbital position as a function of time. This is the classic two-body problem, solved in any dynamics textbook \citep[e.g.][]{murray99}. In addition to the orbital solution, the transit model also requires the planet-to-star radius ratio $R_p/R_s$ and the stellar brightness profile. Due to limb darkening, the stellar disk is brighter at the center than at the edge, so the fraction of flux blocked by the planet depends on its exact position.  \cite{mandel02} and \cite{gimenez06} calculated analytic solutions for the transit light curve for a range of stellar brightness profiles,  which are implemented in a number of publicly available software packages including \texttt{JKTEBOP}, \texttt{TAP}, \texttt{ExoFast}, \texttt{PyTransit}, and \texttt{batman} \citep{southworth04, gazak12, eastman13, parviainen15, kreidberg15a}.  FIXME say more about limb darkening.

%The physical parameters for the orbit are the period $P$, time of inferior conjunction $t_0$, orbital separation $a$, inclination $i$, eccentricity $e$.  The orbital solution can be obtained via Kepler's law \citep{murray99}.
The eclipse model is a more straightforward calculation: 
\begin{equation}
E(t) = 1 + F_p/F_s (1 - \alpha(t)) 
\end{equation}
where $F_p/F_s$ is the planet-to-star flux ratio and $\alpha(t)$ is the fraction of the planet disk eclipsed by the star (easily calculated from the overlapping area of two circles). This model assumes the brightness of the planet is constant, and neglects planetary limb darkening or other variation in temperature/reflectivity. In practice, these variations are only detectable for the highest precision light curves \citep{FIXME}. 

The planet's phase variation is typically fit using a sinusoid with period equal to the planet's orbital period, with the option of adding higher harmonics \citep[e.g.][]{knutson12, stevenson16}. Alternatively, one can fit an ``orange slice" model that assumes constant planet brightness over slices in longitude \citep{knutson07, cowan08}. 
%A third option is to create a physically motivated temperature map and calculate the \citep{zhang16}. 

\subsection{Estimating Parameter Uncertainties}
As for all measurements, correctly estimating the uncertainty on the planet spectrum is as important as the detection itself.  The best approach for determining the uncertainties depends primarily on whether the noise is correlated in time.  

In the case of uncorrelated noise (referred to as ``white" or ``Gaussian" noise), the uncertainties can be estimated with the Markov chain Monte Carlo (MCMC) technique, a Bayesian method that calculates the posterior probability distribution for the model parameters (see \citealt{sivia96} for a general overview of Bayesian methods and \citealt{ford05} for a specific application of MCMC to exoplanets). These fits can be computationally challenging, especially if there are many parameters, correlated uncertainties, or multi-modal solutions. Sophisticated MCMC methods have been developed to address these issues, including differential evolution MCMC \citep{braak06}, or affine-invariant sampling \citep{goodman10}, implemented in the Python package \texttt{emcee} \citep{foremanmackey13}.

Estimating uncertainties is more difficult in the presence of ``red", time-correlated noise. Red noise is easily identified by eye in the residuals of the light curve fit (see Figure FIXME). To quantitatively test for it, one can bin the data points in time. If the noise is white, the residuals are expected to decrease proportionally to the square root of the number of data points per bin. If the binned residuals are larger than expected, red noise is present and needs to be accounted for in the model fit.  A number of strategies have been developed to model red noise, including ``residual permutation bootstrap" (also known as ``prayer bead")  \citep{FIXME}, wavelet-based methods \citep{carter09}, and Gaussian processes \citep{rasmussen05}. Depending on the amplitude of the red noise, it can increase parameter uncertainties by factors of several, so it must be treated carefully.

\subsection{Sources of Error}

\runinhead{Photon Noise}
\hbindex{Photon noise} is the fundamental limit on the precision of a light curve. A star emits $N \pm \sqrt{N}$ photons per unit time. The $\sqrt{N}$ noise arises because each atom in the star emits a photon with some small probability, so the total number emitted per time interval follows a Poisson distribution. The relative error ($\sqrt{N}/N$) decreases as $N$ increases, so bright stars yield more precise light curves.  The best target systems for atmosphere characterization typically have H mag $< 10$ for infrared observations (or V $< 10$ for optical).  

The precision of a light curve is often quantified by how close it is to the photon-limited noise floor. Space-based observations often reach the photon limit \citep{sing11, kreidberg14a, ingalls16}, whereas ground-based observations are typically a factor of a few above it \citep[e.g.][]{bean13}. FIXME say something about rms.

\runinhead{Instrument Systematics}
The ideal detector would record every incident photon.  Modern detectors have come a long way in achieving this goal, but they are not perfect.  In addition to the usual steps in data reduction (bias and dark correction, flatfielding), there are some considerations that are specific to sub-percent precision spectrophotometry needed for atmosphere characterization.

\begin{itemize}
\item{\emph{Charge trapping}. Near-infrared detectors (e.g. \HST/WFC3) have impurities that can trap photoelectrons \citep{smith08}. As the traps fill up exponentially over time, the observed count rate increases by a factor of approximately $1 - R e^{-t/\tau}$, where $R$ and $\tau$ depend on the detector and illumination level.  This same mechanism is responsible for image persistence, the afterglow the appears as charge traps are released. The effect can be corrected with analytic models to photon-limited precision \citep{deming13, line16}. 
}
\item{\emph{Intrapixel effect.} Detector pixels do not have perfectly uniform spatial sensitivity, so the measured flux is correlated with the $X, Y$ position of the centroid of the image. This effect is the dominant systematic error for infrared measurements (e.g. \Spitzer/IRAC). Many techniques have been developed to model intrapixel variation, including \texttt{BLISS} mapping \citep{stevenson12}, pixel-level decorrelation \citep{deming15}, and ICA \citep{morello15}.}
%\item{\emph{Undersampling}.}
\item{\emph{Geometric effects.} Distortion, moving spectrum.}
\end{itemize}

\runinhead{Astrophysical Systematics}
Systematic errors also arise from incorrect models for the stellar or planet flux. 

\begin{itemize}
\item{\emph{Background stars.} Roughly half of stars have one or more bound companions \citep{raghavan10}. If the companion flux is blended with that of the host star, it dilutes the planet signal. Transit and eclipse depths must be multiplied by a correction factor $1 + \beta(\lambda)$ where $\beta$ is the ratio of the background star to host star flux. High contrast imaging is needed to detect close companions, and should be obtained for systems that are targets for atmosphere characterization.}
\item{\emph{Star spots.} To first order, the temperature difference between the stellar photosphere and the spotted region introduces a slope in the planet spectrum. If the spots are cool enough for molecules to form (e.g. water), they can even produce spurious spectral features, as there is no way to distinguish between absorption due to the spot and absorption by the planet atmosphere. The effect from spots can be corrected by multiplying the planet spectrum by a factor $(1-s\times[1-F_{\lambda,\mathrm{spot}}/F_{\lambda,\mathrm{phot}}])^{-1}$, where $s$ is the spot covering fraction and $F_\lambda$ is spectral radiance for the spot or the photosphere \citep{mccullough14}. The spot properties can be estimated based on the amplitude of long-term photometric variability in the stellar lightcurve or from spot crossing during transit \citep[e.g.][]{pont08}.} 
\item{\emph{Stellar activity.} Variations in star spot coverage are also a source of bias in transit depth measurements. To correct transit depths taken at different epochs, one can obtain photometric monitoring of the host star to estimate changes in $s$ and correct the depths with the above scale factor.} 
\item{\emph{Nightside emission from the planet.} For the hottest planets, thermal emission from the nightside may contribute significant flux during the transit. The nightside temperature depends on the planet's heat redistribution and cloud coverage, so can only be measured directly from phase curve observations. To correct for nightside flux, the transit depths should be multiplied be a factor FIXME \citep{kipping10}.} 
\end{itemize}

\section{Major Results from Atmosphere Studies}
In this section we discuss observational highlights, with a focus on transmission and emission spectroscopy.  For more comprehensive reviews, see \citep{crossfield15} and \citep{deming17}. Before diving into results, let us briefly set the stage with some expectations for what we might observe. 

\subsection{Expectations}
We know from the Solar System that atmospheric composition can vary widely from planet to planet. Even so, we can still make educated guesses about possible compositions based on the building blocks of planet formation.  

Planets form in disks of gas and dust surrounding young stars.  The disk is made predominantly of hydrogen and helium. Trace amounts of heavier elements (metals) are also present, the most abundant being oxygen, carbon, and nitrogen \citep{anders89}.  Exactly how planets arise from this initial disk is a topic of considerable complexity \citep{FIXME}, but the end result is that gas and ice giants retain primordial atmospheres made of disk material, whereas lower mass planets have secondary atmospheres formed by outgassing or sublimation of ices \citep{FIXME}.  Given this picture of planet formation, we can narrow the list of expected abundant species to a few: hydrogen, helium, and oxygen/carbon/nitrogen compounds. At planetary temperatures and pressures, the most likely reservoirs for O, C, and N are H$_2$O, CO, CO$_2$, CH$_4$, and NH$_3$.  

The most abundant species are not necessarily the easiest to see, however.  Observability depends on both abundance \emph{and} absorption cross-section. Hydrogen is not a particularly strong absorber (its main spectral feature is Rayleigh scattering in the optical), and helium FIXME FIXME, as a noble gas, has no detectable features whatsoever. Therefore, we would expect planets' spectra to be dominated by features from volatiles.  In addition, there are a few trace species with very large absorption cross sections (e.g., sodium and titanium oxide) that are detectable as well.

%Another lesson to take from the Solar System is that 
%Even though molecular hydrogen is the dominant component of giant planet atmospheres, it is not the dominant absorber at most wavelengths. Lower abundance 
%and Rayleigh scattering in the optical \citep{lecavelier08}.
%In the core accretion model of planet formation, planetesimals made of rock and ice eventually grow massive enough to accrete surrounding disk material \citep{pollack96}
%The planet composition is divided into three main components:  gas (in this context only H/He), ice (e.g., H$_2$O, CH$_4$, CO, CO$_2$, NH$_3$), and rock (iron, silicates).

\subsection{Chemical Composition}

\runinhead{Detections}
Much of exoplanet atmosphere characterization to date has focused on basic inventory: what atoms and molecules are present?  Several absorbers have been unambiguously identified so far, including H$_2$O, CO, Na, FIXME. 

The most common is H$_2$O detections from \HST/WFC3.

There is also evidence for CO and CO$_2$ from \Spitzer\ photometry. However, the \Spitzer\ bandpasses are about $1\,\mu$m wide, and cover spectral features from multiple different absorbing species (CO, CO$_2$, H$_2$O). It is possible to   

\runinhead{Absolute Abundances and Metallicity}

\runinhead{Abundance Ratios}

\runinhead{Disequilibrium}

\subsection{Atmospheric Escape}

\subsection{Climate}
Temperature pressure profile



\runinhead{Winds}

\subsection{Aerosols}
aerosols -- an umbrella term for clouds and hazes -- are ubiquitous in the Solar System and are proving to be common in exoplanets too. Evidence for aerosols shows up in both transmission and reflection spectra.

Transmission spectra are especially sensitive to the presence of clouds and hazes due to the slant viewing geometry during transit (FIXME: fig?) \citep{fortney05}. The optical path of photons is long, so even trace species have large optical depth relative to a face-on view. 

- weaker than expected features in transmission
high albedos (shiny)
scattering slope in optical transmission spectra

\subsection{Super Earths?}

\section{Future Prospects}
We are on the threshold of a revolution in exoplanet atmosphere characterization thanks to several upcoming observing facilities. 

The first of these is the Transiting Exoplanet Survey Satellite (\TESS), a planet finding mission scheduled to launch in 2018 \citep{ricker14}.  The goal of \TESS\ is to search the 200,000 brightest stars in the sky for transiting planets, with an expected yield of nearly 2000 discoveries \citep{sullivan15}. In contrast to the majority of transiting planets discovered to date by \Kepler \citep{}, the \TESS\ planets will have bright enough host stars for precise atmosphere characterization. The sample will also include longer period and smaller planets than are typically discovered by ground-based surveys of bright stars, opening the door to new parameter space. 

The second game-changer is \JWST. The primary technical limitations in atmosphere characterization thus far have been aperture size and wavelength coverage, and \JWST\ offers major improvements on both fronts. It has roughly ten times the collecting area of \HST\, and provides spectroscopy from the optical to the infrared ($0.6 - 28\,\mu$m) \citep{FIXME}. The improvement in precision will make it possible to push down to sub-Neptune and smaller planets, and to study giant planets at high S/N. The expanded wavelength coverage will also enable the first unambiguous detection of many new molecules (including methane, carbon dioxide, and ammonia). Finally, \JWST's infrared detector will be sensitive to thermal emission from cooler objects, including potentially habitable worlds.

%There are also major advances planned for ground-based observing facilities. CRIRES+ 
% The next generation of large, 30 m class telescopes
%CRIRES +
%noise floor

\runinhead{Earth-Like Planets}
The nearest transiting, Earth-like planet is predicted to be 12 parsecs away \citep{dressing16}.


A recent exciting discovery is Proxima b, a likely terrestrial planet recieving 70\% Earth insolation \citep{anglada16}. 

\section{Bibliography}

\begin{acknowledgement}
FIXME
\end{acknowledgement}

\bibliographystyle{spbasicHBexo}  %for bibtex
\bibliography{chapter} %for bibtex-example

\end{document}
