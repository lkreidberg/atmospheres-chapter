%%%%%%%%%%%%%%%%%%%% author.tex %%%%%%%%%%%%%%%%%%%%%%%%%%%%%%%%%%%
%
% template for chapters to the Handbook of Exoplanets
% modified by H. Deeg from the 'template.tex' provided by Springer for the svmult.cls class
% 20Mar 2016
%
%%%%%%%%%%%%%%%% Springer %%%%%%%%%%%%%%%%%%%%%%%%%%%%%%%%%%


% RECOMMENDED %%%%%%%%%%%%%%%%%%%%%%%%%%%%%%%%%%%%%%%%%%%%%%%%%%%
\documentclass[graybox,natbib,nosecnum]{svmult}
\bibpunct{(}{)}{;}{a}{}{,} % suppress commas between author-names and year

\pdfoutput=1   %forces use of pdflatex. Disable if you prefer to use .eps or .ps figures.
% choose options for [] as required from the list
% in the Reference Guide

\usepackage{mathptmx}       % selects Times Roman as basic font
\usepackage{helvet}         % selects Helvetica as sans-serif font
\usepackage{courier}        % selects Courier as typewriter font
\usepackage{type1cm}        % activate if the above 3 fonts are
                            % not available on your system

\usepackage{makeidx}         % allows index generation
\usepackage{graphicx}        % standard LaTeX graphics tool
                             % when including figure files
\usepackage{multicol}        % used for the two-column index
\usepackage[bottom]{footmisc}% places footnotes at page bottom
\usepackage[normalem]{ulem}	% for strike-through of text with \sout{}  
\usepackage{hyperref}  %for hyperlinks

\usepackage{soul}   % for high-lighting of text
% see the list of further useful packages
% in the Reference Guide

% expansions of  journal abbreviations from bibtex entries by ADS
% adapted to Springer Basic style (no periods in abbreviations)
\newcommand*\aap{A\&A}
\let\astap=\aap
\newcommand*\aapr{A\&A Rev}
\newcommand*\aaps{A\&AS}
\newcommand*\actaa{Acta Astron}
\newcommand*\aj{AJ}
\newcommand*\ao{Appl Opt}
\let\applopt\ao
\newcommand*\apj{ApJ}
\newcommand*\apjl{ApJ}
\let\apjlett\apjl
\newcommand*\apjs{ApJS}
\let\apjsupp\apjs
\newcommand*\aplett{Astrophys Lett}
\newcommand*\apspr{Astrophys Space Phys Res}
\newcommand*\apss{Ap\&SS}
\newcommand*\araa{ARA\&A}
\newcommand*\azh{AZh}
\newcommand*\baas{BAAS}
\newcommand*\bac{Bull astr Inst Czechosl}
\newcommand*\bain{Bull Astron Inst Netherlands}
\newcommand*\caa{Chinese Astron Astrophys}
\newcommand*\cjaa{Chinese J Astron Astrophys}
\newcommand*\fcp{Fund Cosmic Phys}
\newcommand*\gca{Geochim Cosmochim Acta}
\newcommand*\grl{Geophys Res Lett}
\newcommand*\iaucirc{IAU Circ}
\newcommand*\icarus{Icarus}
\newcommand*\jcap{J Cosmology Astropart Phys}
\newcommand*\jcp{J Chem Phys}
\newcommand*\jgr{J Geophys Res}
\newcommand*\jqsrt{J Quant Spectr Rad Transf}
\newcommand*\jrasc{JRASC}
\newcommand*\memras{MmRAS}
\newcommand*\memsai{Mem Soc Astron Italiana}
\newcommand*\mnras{MNRAS}
\newcommand*\na{New A}
\newcommand*\nar{New A Rev}
\newcommand*\nat{Nature}
\newcommand*\nphysa{Nucl Phys A}
\newcommand*\pasa{PASA}
\newcommand*\pasj{PASJ}
\newcommand*\pasp{PASP}
\newcommand*\physrep{Phys Rep}
\newcommand*\physscr{Phys Scr}
\newcommand*\planss{Planet Space Sci}
\newcommand*\pra{Phys Rev A}
\newcommand*\prb{Phys Rev B}
\newcommand*\prc{Phys Rev C}
\newcommand*\prd{Phys Rev D}
\newcommand*\pre{Phys Rev E}
\newcommand*\prl{Phys Rev Lett}
\newcommand*\procspie{Proc SPIE}
\newcommand*\qjras{QJRAS}
\newcommand*\rmxaa{Rev Mexicana Astron Astrofis}
\newcommand*\skytel{S\&T}
\newcommand*\solphys{Sol Phys}
\newcommand*\sovast{Soviet Ast}
\newcommand*\ssr{Space Sci Rev}
\newcommand*\zap{ZAp}


\newcommand{\hbindex}[1]{\hl{#1}\index{#1}}  %highlights index entries

\makeindex             % used for the subject index
                       % please use the style svind.ist with
                       % your makeindex program

%%%%%%%%%%%%%%%%%%%%%%%%%%%%%%%%%%%%%%%%%%%%%%%%%%%%%%%%%%%%%%%%%%%%%%%%%%%%%%%%%%%%%%%%%

\begin{document}

\title*{Template for a Chapter of the Handbook of Extrasolar Planets}
% Use \titlerunning{Short Title} for an abbreviated version of
% your contribution title if the original one is too long
\author{Hans J. Deeg  and First (Middle) Family name of second author}
% Use \authorrunning{Short Title} for an abbreviated version of
% your contribution title if the original one is too long
\institute{Hans J. Deeg \at Instituto de Astrof\'\i sica de Canarias, C. Via Lactea S/N, E-38205 La Laguna, Tenerife, Spain, \email{hdeeg@iac.es}
\and First (Middle) Family name of second author \at Name, Address of Institute \email{name@email.address}}
%
% Use the package "url.sty" to avoid
% problems with special characters
% used in your e-mail or web address
%
\maketitle


\abstract{This document is intended as a template and guide for the preparation of chapters for the Handbook of Exoplanets, using latex. It complements the `Guidelines for Authors of the Exoplanet Handbook', which are distributed as a separate document. This template was adapted from Springer's template `author.tex' for contributed books, using the style svmult.cls (distributed together with this template). \\ Each chapter should be preceded by an abstract (10--15 lines long) that summarizes the content. The abstract will appear \textit{online} at \url{www.SpringerLink.com} and at ADS and be available with unrestricted access. This allows unregistered users to read the abstract as a teaser for the complete chapter. For the title you are encouraged to follow the tips for Search Engine Optimization given below.}

\section{Introduction }
This document is intended as a template and guide for the preparation of chapters for the Handbook of Exoplanets, using latex. As a standard, your contribution should be of 5000-7000 words (or as agreed with the section editors), including references. In the layout of this template, this corresponds to  10-15 pages without figures; e.g. $\approx$ 500 words per page. Contributions should in general follow the usual scheme, "Introduction, Main text (split into various sections with heads and subheads chosen by authors), Conclusions", although circumstances might indicate a deviation from this. The main text is followed by cross-references, optional acknowledgements and the bibliography, as shown in this template. The bibliography may be included within the main .tex document, or be generated using bibtex; for details see the corresponding section. You should also mark \hbindex{relevant terms} for the Handbook's index, as shown in the corresponding section.
{\bf Footnotes should not be used}!

\section{Support}
For issues related to the latex processing, please contact \href {mailto:TeXSupport@spi-global.com}{\ul{TeXSupport@spi-global.com}} (address is clickeable), indicating that you are an author for the Handbook of Exoplanets. For aspects relating to format and structure and general queries, please contact your Development Editor for the project at Springer, Dr. Juby George, \href{mailto:juby.george@springer.com}{\ul{E-mail: juby.george@springer.com}}. Tel: +91 11 4575 5836 | Fax: +91 11 4575 5889. For questions regarding the scientific content, please contact your \href{https://meteor.springer.com/exoplanets/?id=435&tab=About&mode=ReadPage&entity=1870}{\ul{Section Editor}} or the \href{https://meteor.springer.com/exoplanets/?id=435&tab=About&mode=ReadPage&entity=1359}{\ul{Editors in Chief}}.

\section{Compiling this Document with/without bibtex and Submission}
This template has been set-up for use with latex and bibtex, as we expect that most authors will use bibtex.  If a compilation is done from command line, the sequence would be `latex, bibtex, latex, latex'. The compiled document, named HBexoTemplatePdf.pdf, is also distributed together with this template.

You may also compile this document \emph{without} bibtex after in/out-commenting several lines of the bibliography; see the instructions at the end of the the .tex source file.  

\runinhead{Submission of your chapter} To submit, login at \url{http://meteor.springer.com/exoplanets} with the user/password you should have received from Springer. Please name your manuscript files using your last name, e.g. \emph{ackermann.tex, ackermann.bbl}. If you provide several chapters, include a reference to their title, with names like \emph{ackermann\textunderscore transit.tex}. See also the figure naming scheme given below. Please upload the source files required for compilation (.tex, figures, and .bbl if you use bibtex) as well as a {\bf .pdf file of the compiled document}.

\section{The Title of your Chapter - Search Engine Optimization} 
For search engines, but for users as well, the title is the first impression of a chapter or entry. It is the most valuable element for search engines to determine the relevance of an article on a search query. Therefore, it is important to use the right keywords in titles in order to provide a good description of the content of a chapter or entry. Titles shouldn't be too short or too long. Please note that titles for major reference works entries differ greatly from titles of journal articles.

\runinhead{Guidelines for a good title:}
\begin{itemize}
\item The \ul{ first word} (or first few words) \ul{of a title should always be a keyword}. Preferably, start the title with the most descriptive and specific keywords, as words at the beginning of a title will get more value. Google only shows the first few words. 
\item Use the \ul{most important keywords within the first 65 characters}. Only the first 65 characters (including spaces) will be shown in Google. 
\item Limit articles (the, a, an) and punctuation (especially question marks, parentheses, slashes). 
\item \ul{Title entries as if they are standalone} because most readers will search online using keywords and so will access individual chapters directly. \end{itemize}

\runinhead{Example for Handbook titles:}
\begin{itemize}
\item {Titles should not be too short, too long, or too creative, especially at the sake of lacking keywords. For example, ``When the Sun Rises, I'm Able to Sleep -- A Study of Maturing Geographies of Sleep among Young Brazilians on the Street" is too creative and too long. A better title would be ``Geographies of Sleep among Brazilian Street Youth."}
\item ``Evaluation and Management of Pediatric Phalanx Fractures" should be ``Pediatric Phalanx Fractures: Evaluation and Management."
\end{itemize}

\section{Sections and further  Sub-Divisions}
Instead of simply listing headings of different levels we recommend to
let every heading be followed by at least a short passage of text, as shown in the following structure.

\subsection{Subsection Heading}
Example of a subsection. Instead of simply listing headings of different levels we recommend to
let every heading be followed by at least a short passage of text.

\runinhead{Run-in Heading Boldface Version} This is a bold-faced run-in header followed by normal text, created with \textbackslash runinhead\{\}.

\section{Equations}
Use the standard \verb|equation| environment to typeset your equations, e.g.
%
\begin{equation}
a \times b = c\;,
\end{equation}
%
however, for multiline equations we recommend to use the \verb|eqnarray| environment:
\begin{eqnarray}
a \times b = c \nonumber\\
\vec{a} \cdot \vec{b}=\vec{c}
\label{eq:01}
\end{eqnarray}
The second line also shows how vectors should be written.

\section{Lists}
For typesetting numbered lists we recommend to use the \verb|enumerate| environment -- it will automatically render Springer's preferred layout.

\begin{enumerate}
\item{Livelihood and survival mobility are oftentimes coutcomes of uneven socioeconomic development.}
\begin{enumerate}
\item{Livelihood and survival mobility are oftentimes coutcomes of uneven socioeconomic development.}
\item{Livelihood and survival mobility are oftentimes coutcomes of uneven socioeconomic development.}
\end{enumerate}
\item{Livelihood and survival mobility are oftentimes coutcomes of uneven socioeconomic development.}
\end{enumerate}

For unnumbered list we recommend to use the \verb|itemize| environment -- it will automatically render Springer's preferred layout.

\begin{itemize}
\item{Livelihood and survival mobility are oftentimes coutcomes of uneven socioeconomic development, cf. Table~\ref{tab:1}.}
\begin{itemize}
\item{Livelihood and survival mobility are oftentimes coutcomes of uneven socioeconomic development.}
\item{Livelihood and survival mobility are oftentimes coutcomes of uneven socioeconomic development.}
\end{itemize}
\item{Livelihood and survival mobility are oftentimes coutcomes of uneven socioeconomic development.}
\end{itemize}




\section{Figures}
Color figures can be used throughout the handbook; for general information about figures, see the separate document "Guidelines for Authors".

To insert figures, the graphicx package (as done in this template) should be used. You may provide figures either as  .png, .jpg, .pdf files for processing with pdfTeX/tex (default) OR as .eps or .ps files (for processing with the traditional dvips). The previous OR is exclusive, so if you use .eps or .ps, you need to use it for all figures!  Also, if you use .eps or .ps figures, you must remove the \textbackslash pdfoutput=1 command at the begin of this template.

Other graphics packages can be used if they are part of the standard TeX distribution -- if in doubt contact Springer's latex support (see section on support ). 

Please name your figures by using your last name and the figure number, e.g., ackermann\textunderscore fig1.pdf and ackermann\textunderscore fig2.jpg. If you provide several chapters, use names like ackermann\textunderscore transit\textunderscore fig.1.pdf.
 
An example of a figure is shown in Fig.~\ref{fig:1}. Please use the \LaTeX\ automatism for referencing figures. You do not need to worry about layout, as this will be done during the editing. 


% For figures use
\begin{figure}
% Use the relevant command for your figure-insertion program
% to insert the figure file.
% For example, with the graphicx style use
\includegraphics[scale=.65]{template_fig1}
%
\caption{Example figure. You do not have to worry on the layout as this will be revised by Springer}
\label{fig:1}       % Give a unique label
\end{figure}

\section{Tables}
Please use a format and style similar to major astronomy journals. Above the table, give only its title. More details can be given in a caption or footnote below the table. Please use the \LaTeX\ automatism for referencing tables. Table~\ref{tab:1} is an example table.


% Use the \index{} command to code your index words
%
% For tables use
%
\begin{table}
\caption{Please write your table caption here.}
\label{tab:1}       % Give a unique label
%
% Follow this input for your own table layout
%
\begin{tabular}{p{2cm}p{2.4cm}p{2cm}p{4.9cm}}
\hline\noalign{\smallskip}
Classes & Subclass & Length & Action Mechanism  \\
\noalign{\smallskip}\svhline\noalign{\smallskip}
Translation & mRNA$^a$  & 22 (19--25) & Translation repression, mRNA cleavage\\
Translation & mRNA cleavage & 21 & mRNA cleavage\\
Translation & mRNA  & 21--22 & mRNA cleavage\\
Translation & mRNA  & 24--26 & Histone and DNA Modification\\
\noalign{\smallskip}\hline\noalign{\smallskip}
\end{tabular}
Give details in a table foot note. $^a$ This is a comment to an entry in the table
\end{table}
%

\section{Other Formatting Options}
Some other formatting options are given here.
\begin{svgraybox}
If you want to emphasize complete paragraphs of texts we recommend to use the newly defined Springer class option and environment \verb|svgraybox|. This will produce a 15 percent screened box `behind' your text.
\end{svgraybox}

If you want to list definitions or the like we recommend to use the Springer-enhanced \verb|description| environment -- it will automatically render Springer's preferred layout.
\begin{description}[Type 1]
\item[Type 1]{This is the content of the description.}
\item[Type 2]{Another content of a description.}
\end{description}

\section{Hyperlinks}
If you mention a web-source that you consider unsuitable for inclusion into the references, please use the {\bf \textbackslash url\{http:\/\/xxx.xx.xx\}} command and write out the full url. Do \emph{not} hide the url behind a descriptive text (e.g. via the \textbackslash href command), as it must be visible in the printed edition (This guide-document ignores this advice, but neither is it intended for printing) . As an example, the web-site of the Handbook of Exoplanets is found at \url{http://refworks.springer.com/mrw/index.php?id=7494}. This link is clickeable.

\section{Marking Words for the Index}
About 2 -3 terms per page should be marked for the index.  For this, use the {\bf \textbackslash hbindex\{term\}} command. This command highlights the terms that are being indexed. Please do \emph{not use} the normal \textbackslash index\{\} command, since editors need to be able to locate indexed entries in the compiled document.  Here is an example of an \hbindex{indexed term}.

\section{Citations in Text}
Citations are in author-year style using natbib citation commands like {\bf \textbackslash citep\{\}} and {\bf \textbackslash citet\{\}}. The basic \textbackslash cite command works identical to \textbackslash citet. For more details on these commands, see e.g. \href{http://osl.ugr.es/CTAN/macros/latex/contrib/natbib/natnotes.pdf} {(\ul{clickeable link})}.

Here some example citations:\\
\cite{akaike74}  \textbackslash cite\\
\citet{akaike74} \textbackslash citet\\
\citep{akaike74} \textbackslash citep\\
With these commands, works with 2 authors are automatically cited with both authors, like \citep{Boisnard06}, and works with 3 or more authors will appear with `et al', like \citep{2013A&A...550A..67P}. Citations of multiple works are being separated by a semi-colon; e.g. \citep{akaike74, giclas+71}.

\section{Bibliography}
References use the `Springer Basic Style'. Examples for different types of works are given in \href{https://meteor.springer.com/exoplanets/?id=435&tab=About&mode=ReadPage&entity=3283}{\ul{Springer's syle sheet (clickable link)}}. 
\noindent Please note the following on the reference-style:
\begin{itemize}
\item{Initials of first and middle names are concatenated without space, e.g Smith, John Francis appears as Smith JF}
\item{Separate authors by commas; \emph{no} ampersands are used.}
\item{If there are {\bf more then 5 authors}, then {\bf list the first 3 and et al.'}}
\item{{\bf Include titles} (possibly shortened ones), {\bf also for journal articles}!  }
\item{Use journal abbreviations at least for these major journals in Astronomy: A\&A, AJ, ApJ, ARAA, MNRAS, PASP, PASJ. }
\item{Abbreviated Journal names do \emph{not} contain periods. Numerous further journals are defined at the begin of this template. }
\item{The references should be sorted in alphabetical order. If there are several works by the same author, use this order: }
\begin{enumerate}
\item All works by the author alone, ordered chronologically by year of publication
\item All works by the author with a coauthor, ordered alphabetically by coauthor
\item All works by the author with several coauthors, ordered chronologically by year of publication.
\end{enumerate}
\end{itemize}
The bibliography will receive a final editing by Springer, so you do not need to worry on style issues as long as all required information is given.

For the most common types of works, examples are also given here, whose appearance may be consulted in the reference-section: 
\begin{itemize}
\item[-]{Journal paper: \citep{2013A&A...550A..67P} Listing the DOI is optional. }
\item[-]{Journal paper, many authors: \citep{almenara09} }
\item[-]{Book: \citep{all73}  }
\item[-]{Book, 3 authors: \citep{giclas+71}  }
\item[-]{Chapter in book: \citep{2015hae..book.1501B}  \emph{Bibtex users, see comment in section on bibtex.} }
\item[-]{Proceedings:  \citep{Boisnard06}  }
\item[-]{PhD thesis: \citep{AlmenaraThesis10}  }
\end{itemize}

\subsection{bibtex}
The use of bibtex is recommended, taking care of the style issues listed previously. Bibex will use the style file 'spbasicHBexo.bst', distributed together with this template. The  .bib file associated to this template, HBexoTemplatebib.bib', was compiled from ADS with minor modifications.\\

$\bullet$ In bibtex entries sourced from ADS (and possibly also from other sources) , we noted that {\bf replacement of @INBOOK with @INPROCEEDINGS} generates a more complete citation that includes book-title and editors; it also avoids some bibtex-warnings.

\section{Cross-References}
Please keep this as the last section. If you are aware of other chapters in the handbook that are closely related to yours, you may indicate the titles of these chapters, but please not more than 10 of them. If uncertain on the exact title, attempt an approximation so that the editors can insert the exact title later. Editors may also add further chapters here. Please indicate one chapter per line as show in example below. {\bf You should delete these instructions in your contribution.}
\begin{itemize}
\item{Other worlds in ancient greek philosophy}
\item{Exoplanet detection in the 21st century}
\item{Fundamental limits to knowledge on exoplanets}
\end{itemize}

\begin{acknowledgement}
Optionally, include a short acknowledgment.  Else, out-comment this section. Not more than a few lines please.
\end{acknowledgement}

%  IF you do NOT use bibtex, put comments before the following 2 lines
\bibliographystyle{spbasicHBexo}  %for bibtex
\bibliography{HBexoTemplatebib} %for bibtex-example

% IF you do NOT use bibtex, remove comments from all following lines, until \end{document}
%\begin{thebibliography}{}
%\bibitem[Akaike (1974)]{akaike74} Akaike H (1974) A New Look at the Statistical Model Identification. IEEE  Transactions on Automatic Control 19:716--723
%\bibitem[Allen(1973)]{all73} Allen C (1973) Astrophysical Quantities. London: Athlone Press
%\bibitem[Almenara(2010)]{AlmenaraThesis10} Almenara J (2010) Detecci{\'o}n de planetas en sistemas binarios eclipsantes. PhD thesis, Univ. de La Laguna, Tenerife, Spain
%\bibitem[Almenara et~al(2009)]{almenara09} Almenara JM, Deeg HJ, Aigrain S et~al (2009) Rate and nature of false  positives in the CoRoT exoplanet search. \aap 506:337--341, doi 10.1051/0004-6361/200911926, eprint 0908.1172
%\bibitem[Belmonte(2015)]{2015hae..book.1501B} Belmonte JA (2015) Orientation of Egyptian Temples: An Overview. In: Ruggles CLN (ed) Handbook of Archaeoastronomy and Ethnoastronomy, p 1501,  doi {10.1007/978-1-4614-6141-8-146}
%\bibitem[Boisnard and Auvergne(2006)]{Boisnard06}{Boisnard} L {Auvergne} M (2006) {CoRoT in Brief}. In: {Fridlund} M, {Baglin}  A, {Lochard} J {Conroy} L (eds) ESA Special Publication, ESA Special  Publication, vol 1306, p~19
%\bibitem[{{Giclas} et~al(1971){Giclas}, {Burnham}, and {Thomas}}]{giclas+71}{Giclas} HL, {Burnham} R {Thomas} NG (1971) Lowell proper motion survey Northern Hemisphere. The G numbered stars. 8991 stars fainter than magnitude  8 with motions larger than 0.26''/year. Flagstaff, Arizona: Lowell  Observatory, 1971
%\bibitem[{{Parviainen} et~al(2013){Parviainen}, {Deeg}, and {Belmonte}}]{2013A&A...550A..67P} Parviainen H, Deeg HJ Belmonte JA (2013) Secondary eclipses in the CoRoT light curves. A homogeneous search based on Bayesian model selection. \aap  550:A67, doi 10.1051/0004-6361/201220081, eprint 1211.5361
%\end{thebibliography}

\end{document}
